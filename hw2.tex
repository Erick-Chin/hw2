% Created 2024-10-17 Thu 13:29
% Intended LaTeX compiler: pdflatex
\documentclass[11pt]{article}
\usepackage[utf8]{inputenc}
\usepackage[T1]{fontenc}
\usepackage{graphicx}
\usepackage{longtable}
\usepackage{wrapfig}
\usepackage{rotating}
\usepackage[normalem]{ulem}
\usepackage{amsmath}
\usepackage{amssymb}
\usepackage{capt-of}
\usepackage{hyperref}
\usepackage{langsci-avm}
\newcommand{\grmr}[2]{\ensuremath{\mathrm{#1} & \,\longrightarrow\, \mathrm{#2}}}
\newcommand{\txtgrmr}[2]{\ensuremath{\mathrm{#1} \,\longrightarrow\, \mathrm{#2}}}
\newcommand{\grmrhs}[1]{\ensuremath{& \,\longrightarrow\, \mathrm{#1} }}
\newcommand{\wa}[1]{\type{\textnormal{\w{#1}}}}
\author{Toni Kazic}
\date{Fall, 2024}
\title{Homework 2: Formal Languages, Parsing, and Semantics}
\hypersetup{
 pdfauthor={Toni Kazic},
 pdftitle={Homework 2: Formal Languages, Parsing, and Semantics},
 pdfkeywords={},
 pdfsubject={},
 pdfcreator={Emacs 29.3 (Org mode 9.8)}, 
 pdflang={English}}
\begin{document}

\maketitle
\section{Introduction}
\label{sec:org1afeaa0}

In this homework, the syntactic and semantic rubber hits the road.  This
homework introduces the deeper structures of language, especially when
phrased formally; looks at syntax and parsing; and extends the notion of
parsing to semantics.
\section{Who's Who and Solution Patterns}
\label{sec:org7bc4000}
\label{whoswho}
\subsection{Lead Person:  yellow}
\label{sec:orgeea034d}


\subsection{Group Members}
\label{sec:org37e8cf2}

\begin{center}
\begin{tabular}{ll}
first name last name & color\\
\hline
Eric Chin & yellow \color{yellow}\rule{5mm}{3mm}\\
Ying-Chen Lin & green \color{green}\rule{5mm}{3mm}\\
James Wiliams & purple \color{violet}\rule{5mm}{3mm}\\
\end{tabular}
\end{center}
\subsection{Three Member Solution Patterns}
\label{sec:org2f474c7}

\(i\) is the question number.

\begin{center}
\(\begin{array}{ccc}
 \text{color} & \text{draft solution} & \text{revise solution} \\
\hline green \color{green}\rule{5mm}{3mm} & i \mod 3 = 1 & i \mod 3 = 0 \\
 yellow \color{yellow}\rule{5mm}{3mm} & i \mod 3 = 2 & i \mod 3 = 1 \\
 purple \color{violet}\rule{5mm}{3mm} & i \mod 3 = 0 & i \mod 3 = 2 \\
\end{array}
\)
\end{center}
\subsection{Two Member Solution Patterns}
\label{sec:org8953c0c}

\begin{center}
\begin{tabular}{lll}
color & draft solution & revise solution\\
\hline
green \color{green}\rule{5mm}{3mm} & odds & evens\\
yellow \color{yellow}\rule{5mm}{3mm} & evens & odds\\
\end{tabular}
\end{center}
\section{General Instructions}
\label{sec:org8b59493}


\begin{itemize}
\item \emph{Fill out the group members table and follow the solution patterns} in
Section \ref{whoswho}.

\item \emph{If the question is unclear, tell me your interpretation of it as part
of your answer.}  Feel free to ask about the questions in class or on
the Slack channel (use \texttt{@channel} as others will probably be puzzled
too).

\item \emph{For questions using corpora, use the corpus of the lead person.}

\item \emph{Put your draft answers right after each question using a \textbf{complete,
functional} \texttt{org} mode code or example block.} Make sure your code
block is complete and functional by testing it in your copy of this
homework file.

\item \emph{Each group member reviews the others' draft solutions and you revise them together}.

\item \emph{Discuss each other's draft and reviews, finalizing the answers.}

\item \emph{Show all your work: code, results, and analysis.}  Does your code
work in this file and produce \textbf{exactly} the results you show?

\item \emph{Post the completed file to Canvas no later than noon on the Tuesday
indicated} in the \href{../syllabus.org}{schedule in the syllabus}, naming your file with each
person's first name (no spaces in the file name, and don't forget the
\texttt{.org} extension!).  Only one person should submit the final file.
\end{itemize}
\section{Hints}
\label{sec:org78456b4}


\subsection{Make sure the structure of the grammar can be parsed by the parser.}
\label{sec:orgf9dddc5}

For example, a recursive descent parser cannot terminate the parse of a
left-recursive grammar.
\subsection{Use re-entrancy if you need it.}
\label{sec:orgc8574ac}

\texttt{NLTK} has some notation for this.
\section{Questions}
\label{sec:org4e774d6}

\begin{enumerate}
\setcounter{enumi}{0}
\item \label{prod-rules} Remember those silly tags from \href{./hw1.org}{hw1.org}?  Let
\end{enumerate}

\begin{align}
N &= \{ \textrm{FOO,BAR,EGO,NEED,ADS,DUCK,MANSE} \} \ \text{and} \nonumber \\
T &= \{ \w{dog, black, racing, was, squirrel, tree, burrow, ground hog, bushes, towards,
hunting, back, wee} \}  \nonumber
\end{align}

For each of the following production rules, state from which class of
language they come and \emph{show why} by derivation from the language
definitions (that is, write the proof and describe it).
\begin{description}
\item[{rule 1}] \(\txtgrmr{FOO}{EGO \ NEED \ DUCK}\)
\begin{description}
\item[{ANSWER}] This is a production of a Context Free Language.
\begin{itemize}
\item This is because the rule can be rewritten as \(A \rightarrow BCD\) where \(B,C,D \in N\)
\item \(BCD \in w\) given that \(w \in \{N,T\}^*\)
\end{itemize}
\end{description}
\item[{rule 2}] \(\txtgrmr{FOO \ DUCK}{EGO \ NEED \ DUCK}\)
\begin{description}
\item[{ANSWER}] This is a Context Sensitive Language production!
\begin{itemize}
\item This is because this production is of the form \(u_1 A u_2 \rightarrow u_1 w u_2\)
\item \(\epsilon = u_1, FOO = A, DUCK = u_2, EGO\ NEED = w\)
\item The following holds true: \(u_1, u_2, w \in (N \cup T)^*, A \in N, w \neq \epsilon\)
\end{itemize}
\end{description}
\end{description}


\begin{description}
\item[{rule 3}] \(\txtgrmr{FOO}{EGO \ \w{dog} \ DUCK}\)
\begin{description}
\item[{ANSWER}] This is a production of a Context Free Language.
\begin{itemize}
\item This is because the rule can be rewritten as \(A \rightarrow BCD\) where \(B,D \in N\) and \(C \in T\)
\item \(BCD \in w\) given that \(w \in \{N,T\}^*\)
\end{itemize}
\end{description}
\item[{rule 4}] \(\txtgrmr{FOO \ \w{ground hog}}{EGO \ \w{dog} \ DUCK \ \w{squirrel}}\)
\begin{description}
\item[{ANSWER}] 
\end{description}
\item[{rule 5}] \(\txtgrmr{FOO}{\w{black} \ \w{dog} \ DUCK}\)
\begin{description}
\item[{ANSWER}] This is a Regular Language!
\begin{itemize}
\item This is because the rule can be redescribed as \(A \rightarrow XY\) where \(X \in T^*\) and \(Y \in N\)
\end{itemize}
\end{description}
\end{description}

Each of the rules is abstracted from a different grammar!



\begin{enumerate}
\setcounter{enumi}{1}
\item \label{which-recursn} Consider the following grammar.
\end{enumerate}
\begin{verbatim}
N = {A,B,C,D}
T = {foo,bar}
S = {C}
P = {C -> A B
     B -> A D
     B -> A
     D -> A A
     A -> T
     }
\end{verbatim}
Is the grammar left-, right-, neither-, or both-recursive?  Why?

\begin{enumerate}
\setcounter{enumi}{2}
\item \label{manual-parse} By hand, generate a construct for each unique
length of output using the grammar of question \ref{which-recursn} and show
them and their derivation as an \texttt{org} table:
\end{enumerate}

\begin{center}
\begin{tabular}{ll}
sentence & rule sequence and comments\\
\hline
\(foobar\) & \(S \rightarrow C \rightarrow AB \rightarrow AA \rightarrow AT \rightarrow TT \rightarrow "foo"T \rightarrow "foo"\ "bar"\)\\
\(foofoofoofoo\) & \(S \rightarrow C \rightarrow AB \rightarrow TB \rightarrow "foo"B \rightarrow "foo"AD \rightarrow "foo"TD \rightarrow\)\\
 & \("foo""foo"D \rightarrow "foo""foo"AA \rightarrow "foo""foo"TA \rightarrow\)\\
 & \("foo""foo""foo"A \rightarrow "foo""foo""foo"T \rightarrow "foo""foo""foo""foo"\)\\
\end{tabular}
\end{center}


\begin{enumerate}
\setcounter{enumi}{3}
\item \label{regex} For the terminals in question \ref{prod-rules}, write the
minimum number of Python \textbf{regular expressions} (\emph{not production rules
from context-free or greater grammars!}) to distinguish among them \emph{when
the entire group of words is presented} (not one-by-one!).  Do not use
trivial regexes that match one and only one word.  Use conditionals and
order the regexs into a tree until all terminals are recognized without
ambiguities.  Carry your tree out until each terminal has a regular
expression that places it in the leaves.  Include a sketch of your tree
if you think it will help!

\setcounter{enumi}{4}
\item \label{regex-imp} Now implement your regular expression tree in
question \ref{regex} and show the code and results.
\end{enumerate}



\begin{enumerate}
\setcounter{enumi}{5}
\item \label{first-gram} Write a grammar that captures the following sentences:
\end{enumerate}

\begin{description}
\item[{sentence 1}] ``The cheerful black dog slept quietly by the chair.''
\item[{sentence 2}] ``A sleepy yellow dog stretched his back.''
\item[{sentence 3}] ``Somebody downstairs made the coffee.''
\item[{ANSWER}] Productions Below
\begin{itemize}
\item \(S \rightarrow NP\ VP\)
\item \(NP \rightarrow Det\ ADJP\ Noun\)
\item \(NP \rightarrow Det\ Noun\)
\item \(NP \rightarrow Noun\ Adj\)
\item \(ADJP \rightarrow Adj\ ADJP \|\ Adj\)
\item \(VP \rightarrow Verb\ Adv\ PP\)
\item \(VP \rightarrow Verb\ NP\)
\item \(PP \rightarrow Prep\ NP\)
\item \(Det \rightarrow "the" \| "a" \| "his"\)
\item \(Noun \rightarrow "dog" \| "chair" \| "back" \| "somebody" \| "coffee"\)
\item \(Adj \rightarrow "cheerful" \| "black" \| "sleepy" \| "yellow" \| "downstairs"\)
\item \(Verb \rightarrow "slept" \| "stretched" \| "made"\)
\item \(Prep \rightarrow "by"\)
\item \(Adv \rightarrow "quietly"\)
\end{itemize}
\end{description}

Put the phrases generated by each rule from these sentences alongside the
rules, again as an \texttt{org} table (this example is \textbf{JUST to illustrate format,
it is not correct!}):

\begin{center}
\begin{tabular}{ll}
rule & phrase\\
\hline
S \(\rightarrow\) NP VP & (The cheerful black dog slept quietly by the chair)\\
S \(\rightarrow\) NP VP & (A sleepy yellow dog stretched his back)\\
S \(\rightarrow\) NP VP & (Somebody downstairs made the coffee)\\
NP \(\rightarrow\) Det ADJP Noun & (The cheerful black dog)\\
NP \(\rightarrow\) Det ADJP Noun & (A sleepy yellow dog)\\
NP \(\rightarrow\) Det Noun & (the coffee)\\
NP \(\rightarrow\) Det Noun & (his back)\\
NP \(\rightarrow\) Noun Adj & (Somebody downstairs)\\
ADJP \(\rightarrow\) Adj ADJP & (sleepy yellow)\\
ADJP \(\rightarrow\) Adj ADJP & (cheerful  black)\\
ADJP \(\rightarrow\) Adj & (downstairs)\\
ADJP \(\rightarrow\) Adj & (sleepy)\\
ADJP \(\rightarrow\) Adj & (yellow)\\
ADJP \(\rightarrow\) Adj & (cheerful)\\
ADJP \(\rightarrow\) Adj & (black)\\
VP \(\rightarrow\) Verb Adv PP & (slept quietly by the chair)\\
VP \(\rightarrow\) Verb NP & (stretched his back)\\
VP \(\rightarrow\) Verb NP & (made the coffee)\\
PP \(\rightarrow\) Prop NP & (by the chair)\\
\end{tabular}
\end{center}


\begin{enumerate}
\setcounter{enumi}{6}
\item \label{recur-desc-parser} Now implement the grammar of question
\ref{first-gram} as a recursive descent parser.  Parse each sentence, showing
the results as a prettily printed tree, and compare them.  What do you
observe?

\item \label{chart-parser} Following on, implement the grammar of question
\ref{first-gram} as a chart parser.  Parse each sentence, showing the results,
and compare these chart parsing results to your results in question
\ref{recur-desc-parser}.  What do you observe?

\item \label{clock} Extend your grammar for the sentences in question
\ref{recur-desc-parser}  so that it can parse sentences 4--6 below.  Time the
implementation's performance for each sentence, doing this 1000 times
for each sentence for better estimates, and put the results  in an \texttt{org} table.
\begin{description}
\item[{sentence 4}] ``We had a long walk to the park and Vinny played with
three other dogs.''
\item[{sentence 5}] ``It was sunny today but might not be tomorrow.''
\item[{sentence 6}] ``There are 49 angels dancing on the head of this pin.''
\end{description}

\item \label{avm-graph} Consider this attribute-value matrix:
\end{enumerate}

\begin{center}
\avm{
[ CAT  & s \\
  HEAD & [ AGR   & \1 [ NUM & sg \\
                        PER & 3 ] \\
           SUBJ  & [ AGR \1 ] ] ] \\
}.
\end{center}

Draw the corresponding directed acyclic graph, ideally in Python.  (A hand-drawn figure is fine:
just photograph it and include the image below, as is done in \href{../notes.org}{notes.org}.)

\begin{enumerate}
\setcounter{enumi}{10}
\item \label{basic-fea-struc} Now extend your grammar from question \ref{clock} to include features
relevant to subject-verb agreement, using \texttt{nltk.FeatStruct()} from
chapter nine, so that you can parse sentences 1--9.  Using
\texttt{cp.parse()}, print and study the parse trees for each sentence.  Do
you agree with them?  Why or why not?

\begin{description}
\item[{sentence 7}] ``The black dogs are playing with the elf toy.''
\item[{sentence 8}] ``The yellow dog slept in my pajamas.''
\item[{sentence 9}] ``We will take two long rides in the country next week.''
\end{description}

\setcounter{enumi}{11}
\item \label{fopc} \href{../../reading/blk\_2nd\_ed.pdf}{Chapter 10 of BLK} and \href{http://www.nltk.org/howto/semantics.html}{the semantics howto} march one through the basics
of applying the FOPC and the \(\lambda\) calculus to reifying the semantics of
context-free sentences.  One of the practical difficulties in this
approach is ensuring that the implementation of the universe of
discourse (they call it the \emph{domain of discourse}, same thing) actually
covers the intended universe.

To see this, let's use their \texttt{sem2.fcfg} grammar to parse the following
sentences syntactically and semantically, and output the reification of
the sentences into the FOPC and the \(\lambda\) calculus.  

(HINT: be sure to 
\begin{verbatim}
   from nltk.sem import *
\end{verbatim}
so you get all the parts and save yourself frustration!)  

For each of the following sentences, parse them and print the sentence,
its parse, and its semantics; and then explain the results you get and
exactly how you would fix the problems encountered.

\begin{itemize}
\item Suzie sees Noosa.
\item Fido barks.
\item Tess barks.
\end{itemize}
\end{enumerate}
\section{Grading Scale}
\label{sec:org31ff97e}

This homework is worth 15 points.  The grading
scale is:

\begin{center}
\begin{tabular}{lr}
fraction correctly reviewed and answered & points awarded\\
\hline
\(\ge 0.95\) & 15\\
0.90 -- 0.94 & 14\\
0.85 -- 0.89 & 13\\
0.80 -- 0.94 & 12\\
0.75 -- 0.79 & 11\\
0.70 -- 0.74 & 10\\
0.65 -- 0.69 & 9\\
0.60 -- 0.64 & 8\\
0.55 -- 0.59 & 7\\
0.50 -- 0.54 & 6\\
0.45 -- 0.49 & 5\\
0.40 -- 0.44 & 4\\
0.35 -- 0.39 & 3\\
0.30 -- 0.34 & 2\\
0.25 -- 0.29 & 1\\
\(< 0.25\) & 0\\
\end{tabular}
\end{center}
\section{Scoring}
\label{sec:org2f18251}


\begin{center}
\begin{tabular}{rrl}
question & max pts & answer ok?\\
\hline
1 & 1 & \\
2 & 1 & \\
3 & 1 & \\
4 & 1 & \\
5 & 1 & \\
6 & 2 & \\
7 & 1 & \\
8 & 1 & \\
9 & 1 & \\
10 & 1 & \\
11 & 2 & \\
12 & 2 & \\
\hline
total score & 15 & 0\\
percentage &  & 0\\
total points &  & \\
\end{tabular}
\end{center}
\end{document}
